\chapter{Considerações finais}
\label{cha:conclusao}

A elaboração de uma arquitetura comum para o compartilhamento de informações do ativo é essencial para que haja consistência e interoperabilidade entre os membros da cadeia de suprimentos adotando este sistema. Este trabalho abordou dois aspectos da arquitetura de compartilhamento de dados: o aspecto do fluxo de informações e das interfaces de comunicação entre os Componentes I4.0 (\autoref{cha:arquitetura}) e o aspecto dos tipos de informações e submodelos do produto a serem compartilhados pela cadeia de suprimentos (\autoref{cha:ciclo-de-vida}).

O fluxo de informações ao longo da cadeia de suprimentos por meio de \textit{Web Services} foi baseado na arquitetura do RAMI4.0, que atualmente é o principal modelo de arquitetura de referência para a Indústria 4.0.

Para a representação da dinâmica do compartilhamento entre os elos da cadeia, foi utilizada a técnica PFS (\textit{Production Flow Schema}), pela qual os arcos orientados e as atividades descreveram o fluxo informações ao longo das camadas do RAMI4.0 e entre diferentes Componentes 4.0. Com os diagramas PFS foi possível também estabelecer as responsabilidades de cada camada do RAMI4.0 no processo de fornecimento e consumo de serviços como um todo.

Os detalhes de implementação e tecnologias a serem adotadas na comunicação são indiferentes, desde que implementadas as interfaces de comunicação com o AAS. A arquitetura proposta é, portanto, agnóstica com relação a tecnologias. Possíveis implementações da arquitetura de compartilhamento de dados podem ser exploradas em trabalhos futuros como uma prova de conceito.

Com relação ao ciclo de vida do produto na I4.0, foi abordado como o compartilhamento da MDP pode contribuir para a geração de valor por meio da melhoria de projeto do produto e na melhoria operacional do produto. É mostrado como este compartilhamento de informações integra a cadeia de suprimentos e propicia o surgimento de novos modelos de negócio baseado em dados (\textit{data-driven}).

Por fim, foram apresentadas propostas de submodelos e atributos necessários para a integração dos ativos na cadeia de suprimentos. com informações relevantes a todos os membros. O acesso a estes submodelos permite a colaboração entre os membros e abre assim oportunidades para uma tomada de decisões coordenada a fim de minimizar ineficiências inerentes da complexidade de cadeias de suprimentos.

O acesso a estas informações traz a possibilidade de uma comunicação direta entre os elos e a oportunidade de uma tomada de decisões coordenada a fim de remover as ineficiências inerentes das cadeias de suprimentos.

Considerações sobre privacidade de dados e consentimento de compartilhamento de informações por parte do usuário podem ser exploradas em trabalhados futuros, visando adequar a arquitetura às leis e normas locais de proteção de dados de cada país onde a cadeia de suprimento opera.
