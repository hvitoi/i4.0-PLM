\chapter{Considerações finais}
\label{cha:conclusao}

Este trabalho teve como objetivo geral a elaboração de uma arquitetura baseada no RAMI4.0 para o compartilhamento da MDP ao longo da CS.

A definição de uma arquitetura é essencial para que haja consistência e interoperabilidade entre os membros da CS. Este trabalho abordou este objetivo geral por meio dos objetivos específicos detalhados a seguir:

\begin{itemize}
  \item \textbf{Integração da MDP ao C4.0} (\autoref{sec:estrutura-aas}): os termos C4.0 e MDP surgiram sob contextos distintos, portanto, como passo inicial para o compartilhamento da MDP, foi necessário integrá-la ao C4.0 para que assim seja interoperável com outras partes de um sistema produtivo dentro do contexto da I4.0. Desta forma, todos os componentes do sistema são tratados como C4.0 com diferentes papéis (C4.0-Cliente, C4.0-Servidor e C4.0-Repositório);
  \item \textbf{Levantamento de submodelos do produto e suas propriedades} (\autoref{sec:submodelos-produto}): foram levantadas informações relevantes a serem compartilhadas em uma CS. Estas informações foram classificadas e então inseridas como propriedades agrupadas em submodelos no C4.0;
  \item \textbf{Levantamento dos impactos em geração de valor com o compartilhamento da MDP} (\autoref{sec:geracao-de-valor}): foram apresentados os impactos em geração de valor que o compartilhamento dos submodelos propostos trazem. Estes impactos foram classificados como ``melhorias operacionais'' e ``melhorias de projeto''. Também foram feitas considerações sobre como as informações são coletadas ao longo do ciclo de vida do produto, gerando valor continuamente por meio destas melhorias;
  \item \textbf{Detalhamento do componentes e operações da arquitetura} (\autoref{sec:componentes-e-operacoes}): a arquitetura para o compartilhamento da MDP foi apresentada, mesclando os conceitos do RAMI4.0 com a lógica de fornecimento de serviços por meio de WS. Os componentes (C4.0-Cliente, C4.0-Servidor e C4.0-Repositório) e as operações (publicação, busca, interação) foram detalhados;
  \item \textbf{Descrição do processo de compartilhamento de informações} (\autoref{sec:fluxo-de-fornecimento-de-servicos}): o processo de compartilhamento de informações foi exemplificada por meio de dois cenários realísticos (múltiplos clientes interagindo com um produto e cliente único interagindo com múltiplos produtos);
  \item \textbf{Mapeamento das operações para as camadas do RAMI4.0} (\autoref{sec:mapeamento-das-operacoes}): o fluxo de informações foi modelado por meio de diagramas PFS. Foram modelados os três tipos de operação (publicação, busca e interação). A partir da modelagem foi possível visualizar o caminho percorrido pela informação ao longo de todas as camadas do RAMI4.0 para cada C4.0 e a transferência de informação entre os C4.0s por meio de APIs.
\end{itemize}

\section{Trabalhos futuros}

A arquitetura proposta não está atrelada a tecnologias específicas, podendo estas serem designadas e substituídas conforme conveniência da época. Como provas de conceito, implementações desta arquitetura podem ser exploradas em trabalhos futuros. Estas implementações podem então ser aplicadas para casos reais da indústria e estes casos de uso podem retroalimentar o próprio desenvolvimento da arquitetura original.

Para a representação dos processos para o compartilhamento de informações entre os elos da CS, foi utilizada a técnica PFS (\textit{Production Flow Schema}). As atividades apresentadas partem de um alto nível de abstração, porém cada atividade apresentada pode ser ainda refinada sucessivamente de forma a detalhar cada iteração. A partir do sucessivo detalhamento novas características podem surgir, como paralelismos, conflitos, etc. Dados estas novas características é possível analisar aspectos dinâmicos do sistema. Esta análise pode ser explorada em trabalhos futuros a fim de analisar as propriedades da rede e identificar eventuais \textit{deadlocks}.

A arquitetura proposta adota algumas simplificações, sendo uma delas a consideração das comunicações somente para os casos de sucesso. Desta forma, não são apontadas eventuais exceções ocorridas em tempo de execução como, por exemplo, quando uma informação do produto não for encontrada ou estiver corrompida. Estes e diversos outros cenários de exceção podem ser explorados em revisões futuras da arquitetura para o seu refinamento.

Por fim, considerações sobre privacidade de dados e consentimento do compartilhamento de informações não foram abordadas neste trabalho, porém esta é uma crescente preocupação, dado o impacto que o vazamento de informações restritas podem acarretar. Adequar a arquitetura às leis e normas locais de proteção de dados de cada país onde a CS opera é um importante aspecto a ser considerado para que a arquitetura seja implementada.
