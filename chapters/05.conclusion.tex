\chapter{Considerações finais}
\label{cha:conclusao}

Este trabalho teve como objetivo geral a elaboração de uma arquitetura baseada no RAMI4.0 para o compartilhamento da memória digital do produto (MDP) ao longo da cadeia de suprimentos.

A definição de uma arquitetura comum é essencial para que haja consistência e interoperabilidade entre os membros da cadeia de suprimentos. Este trabalho abordou este objetivo geral por meio dos objetivos específicos detalhados a seguir:

\begin{itemize}
  \item \textbf{Integração da MDP ao Componente 4.0} (\autoref{sec:estrutura-aas}): os termos ``Indústria I4.0'' (e consequentemente C4.0) e ``Memória Digital do Produto`` surgiram sob contextos distintos, portanto como passo inicial para o compartilhamento da MDP foi necessário integrá-la ao Componente 4.0 para que assim seja interoperável com os outros elementos da arquitetura;
  \item \textbf{Levantamento de submodelos do produto e seus respectivos atributos} (\autoref{sec:submodelos-produto}): foram levantadas informações relevantes a serem compartilhadas em uma cadeia de suprimentos com base em literaturas correlatas. Estas informações foram classificadas e então inseridas como propriedades em submodelos. O submodelo é o agrupamento de informações padrão adotado no RAMI4.0 e necessário para a interoperabilidade dos elementos utilizando este arquitetura comum;
  \item \textbf{Apresentação de considerações sobre a geração de valor por meio do compartilhamento da MDP ao longo ciclo de vida de um produto} (\autoref{sec:geracao-de-valor}): foram apresentados os impactos em geração de valor que o compartilhamento dos submodelos propostos trazem. Estes impactos foram classificados como ``melhorias operacionais'' e ``melhorias de projeto''. Também foram feitas considerações sobre como estas informações são coletadas ao longo do ciclo de vida do produto, gerando continuamente estas duas formas de geração de valor;
  \item \textbf{Apresentação da arquitetura com seus componentes e operações} (\autoref{sec:componentes-e-operacoes}): a arquitetura para o compartilhamento da MDP foi apresentada, mesclando os conceitos do RAMI4.0 com a lógica de fornecimento de serviços utilizando \textit{Web Services}. Os componentes (cliente, servidor e repositório) e as operações (publicação, busca, interação) foram detalhados sob o contexto dos Componentes 4.0;
  \item \textbf{Esquematização da dinâmica de compartilhamento de informações por meio de serviços} (\autoref{sec:fluxo-de-fornecimento-de-servicos}): a dinâmica de compartilhamento de informações foi exemplificada por meio de dois cenários realísticos (múltiplos clientes interagindo com um produto e cliente único interagindo com múltiplos produtos), envolvendo todos os componentes da arquitetura (cliente, servidor e repositório);
  \item \textbf{Modelagem do fluxo de informações dentro das camadas do RAMI4.0} (\autoref{sec:mapeamento-das-operacoes}): o fluxo de informações foi modelado por meio de diagramas PFS. Foram modelados os três tipos de interações da arquitetura (publicação, busca e interação). A partir da modelagem foi possível visualizar o caminho percorrido pela informação ao longo de todas as camadas do RAMI4.0 para cada C4.0 e a transferência de informação entre os componentes por meio de APIs.
\end{itemize}

\section{Trabalhos futuros}

Considerações sobre privacidade de dados e consentimento de compartilhamento não foram abordadas neste trabalho, porém esta é uma crescente preocupação, dado o impacto que o vazamento de informações restritas podem acarretar. Adequar a arquitetura às leis e normas locais de proteção de dados de cada país onde a cadeia de suprimento opera é um importante aspecto a ser considerado para que a arquitetura seja implementada.

A arquitetura proposta não está atrelada a tecnologias específicas, podendo estas serem designadas e substituídas conforme conveniência da época. Como provas de conceitos, implementações desta arquitetura podem ser exploradas em trabalhos futuros. As eventuais dificuldades encontradas nas implementações da arquitetura auxiliam também no refinamento e simplificação da própria arquitetura em forma de novas revisões.

Para a representação da dinâmica do compartilhamento entre os elos da cadeia, foi utilizada a técnica PFS (\textit{Production Flow Schema}). A partir de diagramas PFS é possível derivar redes de petri, com as quais é possível simular o sistema. A simulação do modelo criado pode ser explorado em trabalhados futuros a fim de analisar as propriedades da rede e identificar eventuais \textit{deadlocks}.
