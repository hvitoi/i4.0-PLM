\chapter{Considerações finais}
\label{cha:conclusao}

Este trabalho teve como objetivo geral a elaboração de uma arquitetura baseada no RAMI4.0 para o compartilhamento da MDP ao longo da CS.

A definição de uma arquitetura é essencial para que haja consistência e interoperabilidade entre os membros da CS. Este trabalho abordou este objetivo geral por meio dos objetivos específicos detalhados a seguir:

\begin{itemize}
  \item \textbf{Integração da MDP ao C4.0} (\autoref{sec:estrutura-aas}): os termos ``Indústria I4.0'' (e consequentemente C4.0) e ``Memória Digital do Produto'' surgiram sob contextos distintos, portanto como passo inicial para o compartilhamento da MDP foi necessário integrá-la ao C4.0 para que assim seja interoperável com os outros elementos da arquitetura. Os elementos em seguida foram tratados como C4.0 com diferentes papéis (C4.0-Cliente, C4.0-Servidor e C4.0-Repositório), sendo o produto a fonte das informações;
  \item \textbf{Levantamento de submodelos do produto e suas propriedades} (\autoref{sec:submodelos-produto}): foram levantadas informações relevantes a serem compartilhadas em uma CS com base em literaturas correlatas. Estas informações foram classificadas e então inseridas como propriedades em submodelos. O submodelo é o agrupamento de informações padrão adotado no RAMI4.0 e necessário para a interoperabilidade dos elementos utilizando este arquitetura comum;
  \item \textbf{Levantamento dos impactos em geração de valor com o compartilhamento da MDP} (\autoref{sec:geracao-de-valor}): foram apresentados os impactos em geração de valor que o compartilhamento dos submodelos propostos trazem. Estes impactos foram classificados como ``melhorias operacionais'' e ``melhorias de projeto''. Também foram feitas considerações sobre como estas informações são coletadas ao longo do ciclo de vida do produto, gerando continuamente estas duas formas de valor;
  \item \textbf{Desenvolvimento da arquitetura e detalhamento de seus componentes e operações} (\autoref{sec:componentes-e-operacoes}): a arquitetura para o compartilhamento da MDP foi apresentada, mesclando os conceitos do RAMI4.0 com a lógica de fornecimento de serviços utilizando \textit{Web Services}. Os componentes (C4.0-Cliente, C4.0-Servidor e C4.0-Repositório) e as operações (publicação, busca, interação) foram detalhados sob o contexto dos C4.0;
  \item \textbf{Análise do fluxo de informações fim-a-fim} (\autoref{sec:fluxo-de-fornecimento-de-servicos}): a dinâmica de compartilhamento de informações foi exemplificada por meio de dois cenários realísticos (múltiplos clientes interagindo com um produto e cliente único interagindo com múltiplos produtos), envolvendo todos os componentes da arquitetura;
  \item \textbf{Análise do fluxo de informações entre C4.0s} (\autoref{sec:mapeamento-das-operacoes}): o fluxo de informações foi modelado por meio de diagramas PFS. Foram modelados os três tipos de interações da arquitetura (publicação, busca e interação). A partir da modelagem foi possível visualizar o caminho percorrido pela informação ao longo de todas as camadas do RAMI4.0 para cada C4.0 e a transferência de informação entre os componentes por meio de APIs.
\end{itemize}

\section{Trabalhos futuros}

Considerações sobre privacidade de dados e consentimento do compartilhamento de informações não foram abordadas neste trabalho, porém esta é uma crescente preocupação, dado o impacto que o vazamento de informações restritas podem acarretar. Adequar a arquitetura às leis e normas locais de proteção de dados de cada país onde a CS opera é um importante aspecto a ser considerado para que a arquitetura seja implementada.

A arquitetura proposta não está atrelada a tecnologias específicas, podendo estas serem designadas e substituídas conforme conveniência da época. Como provas de conceito, implementações desta arquitetura podem ser exploradas em trabalhos futuros. Estas implementações podem então ser aplicadas para casos reais da indústria para a validação do modelo. As eventuais dificuldades encontradas nas implementações da arquitetura auxiliam também no refinamento e simplificação da própria arquitetura em forma de novas revisões.

Para a representação da dinâmica do compartilhamento entre os elos da cadeia, foi utilizada a técnica PFS (\textit{Production Flow Schema}). A partir de diagramas PFS é possível derivar redes de Petri, com as quais é possível simular o sistema. A simulação do sistema pode ser explorado em trabalhados futuros a fim de analisar as propriedades da rede e identificar eventuais \textit{deadlocks}.

Por fim, a arquitetura proposta faz algumas simplificações, como a consideração somente dos casos de sucesso. Desta forma, não são abordadas eventuais exceções ocorridas em tempo de execução como, por exemplo, quando uma informação do produto não for encontrada ou estiver corrompida. Estes e diversos outros \textit{corner cases} podem ser explorados em revisões futuras da arquitetura para o refinamento da arquitetura.
