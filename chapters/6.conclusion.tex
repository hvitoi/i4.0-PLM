\chapter{Considerações finais}
\label{cha:conclusao}

A elaboração de uma arquitetura comum para o compartilhamento de informações do ativo é essencial para que haja consistência e interoperabilidade entre os membros da cadeia de suprimentos adotando este sistema.

Este trabalho traça um modelo de compartilhamento de informações por meio de \textit{Web Services} utilizando como arquitetura de base o RAMI4.0, que é atualmente o principal modelo de arquitetura de referência para a Indústria 4.0.

Foi feito o mapeamento das operações de um \textit{Web Service} para o eixo ``Camadas'' do RAMI4.0 por meio de diagramas PFS, permitindo visualizar os fluxos de atividades ocorridas durante todo o processo de consumo da memória digital do produto. Com os diagramas PFS foi possível também estabelecer as responsabilidades de cada camada do RAMI4.0 no processo de fornecimento e consumo de serviços como um todo.

Foi abordado como o compartilhamento da MDP pode contribuir para a geração de valor por meio na melhoria de projeto do produto e na melhoria operacional do produto. É mostrado como este compartilhamento de informações integra a cadeia de suprimentos e propicia o surgimento de novos modelos de negócio baseado em dados (\textit{data-driven}).

Uma visão da MDP sobre o eixo ``Ciclo de Vida e Cadeia de Valor'' do RAMI4.0 é abordada, discutindo melhor o porquê e as atribuições dos AASs como ``tipos'' e como ``instâncias'' nas fases do ciclo de vida do produto.

Por fim, são apresentadas propostas de submodelos e atributos do AAS do C4.0-Servidor (produto) necessários para a integração dos ativos na cadeia de suprimentos com informações relevantes a todos os membros. O acesso a estas informações traz a possibilidade de uma comunicação direta entre os elos e a oportunidade de uma tomada de decisões coordenada a fim de remover as ineficiências inerentes das cadeias de suprimentos.
