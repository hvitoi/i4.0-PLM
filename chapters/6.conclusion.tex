\chapter{Considerações finais}
\label{cha:conclusao}

	A elaboração de uma arquitetura comum para o compartilhamento de informações do ativo é essencial para que haja consistência e interoperabilidade entre os membros da cadeia de suprimentos adotando este sistema.
	
	Este trabalho traça um modelo de compartilhamento de informações por meio de \textit{Web Services} utilizando como arquitetura de base o RAMI4.0, que é atualmente o principal modelo de arquitetura de referência para a Indústria 4.0. 
	
	O mapeamento das operações de um \textit{Web Service} para o eixo ``Camadas'' do RAMI4.0 por meio de diagramas PFS permite a visualização dos fluxos de atividades ocorridas durante todo o processo de consumo da memória digital do produto, assim como estabelece as responsabilidades de cada camada para todo o processo. 
	
	Foi abordado também como a MDP pode contribuir na melhoria de projeto do produto e na eficiência operacional da dinâmica da cadeia de suprimentos e como isso propicia o surgimento de novos modelos de negócio baseado em dados (\textit{data-driven}).
	
	Uma visão da MDP sobre o eixo ``Ciclo de Vida e Cadeia de Valor'' do RAMI4.0 é abordada, discutindo melhor o porquê e as atribuições dos AASs como ``tipos'' e como ``instâncias''.
	
	Por fim, são propostas possíveis informações relevantes à cadeia de suprimentos a serem salvas na MDP indicando quando esta informação devem ser armazenadas, onde e a qual cliente esta informação seria de interesse.
