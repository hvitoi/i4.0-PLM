\chapter{Arquitetura para compartilhamento de informações do ativo}
\label{cha:arquitetura}

A elaboração de uma arquitetura comum para o compartilhamento de informações de um ativo é essencial para que haja consistência e interoperabilidade entre os membros da Cadeia de Suprimentos (CS) adotando este sistema.

Este capítulo tem o objetivo de apresentar detalhes da arquitetura proposta baseada em \textit{Web Services} (WS) nos modelos de uma arquitetura orientada a serviços (SOA) compatível com Componentes I4.0 para o compartilhamento de informações do ativo ao longo da CS. A fim de simplificação do texto, os \textit{Web Services} serão mencionados apenas como ``serviços''.

Neste capítulo é apresentado também o mapeamento dos componentes desta arquitetura dentro do eixo camadas do RAMI4.0.

\section{Componentes e operações dos serviços de AASs}

Os serviços no escopo desta arquitetura são representações das funcionalidades dos Componentes I4.0 e são fornecidos e consumidos entre \textit{Asset Administration Shells} (AAS), ou, de forma mais geral, entre os Componentes I4.0 (C4.0), que engloba o AAS.

A lógica de fornecimento e consumo de serviços proposta para a I4.0 segue os moldes de um \textit{Web Service} detalhado na \autoref{sec:webservices}, onde foram apresentados os componentes e operações (vide \autoref{fig:webservice-componentes}) adaptados ao AAS.

Esta arquitetura envolve três componentes (atores) básicos: O AAS cliente, o AAS servidor e o AAS repositório; e três operações: publicação, busca e interação.

Os serviços disponibilizados remotamente pelo AAS servidor escuta e responde solicitações de clientes por meio de uma determinada rede e porta. Os AAS clientes, por sua vez, consomem o serviço disponibilizado pelo servidor por meio de solicitações.

Nesta seção são apresentados detalhes sobre os componentes e operações necessárias para o fornecimento serviços no mundo conectado da I4.0.

\subsection{Componentes}

Os componentes da arquitetura e suas inter-relações são apresentados na \autoref{fig:aas-ws}.

\begin{figure}[htb]
	\centering
	\includegraphics[width=0.7\textwidth]{aas-ws}
	\caption{Componentes e operações do WS.}
	\label{fig:aas-ws}
\end{figure}

De maneira sucinta, os componentes são descritos da seguinte forma: ``O AAS Servidor'' é a parte que possui um serviço a oferecer para os demais AAS no mundo conectado, o ``AAS Cliente'' é a parte que necessita de um serviço e que age ativamente para receber este serviço e o ``AAS Repositório'' é a parte que armazena informações sobre descrições de diversos serviços, que são disponibilizados na forma de contratos.

A \autoref{tab:componentes-ws} lista os componentes da arquitetura para a I4.0 e suas respectivas descrições detalhadas.

\begin{table}[htb]
	\centering
	\begin{tabular}{p{3cm}p{12cm}}
		\hline
		\textbf{Componente}
		 & \textbf{Descrição}                                                                                                                                                                                                                                                                                                                                                                                                                                                                                                                                                  \\

		\hline
		AAS Servidor
		 & O AAS Servidor é a conexão direta com o ativo. Este AAS extrai as informações sobre seu ativo para sua própria MDP para que assim possam ser disponibilizadas na rede. Cada submodelo do AAS representa um conjunto de informações e serviços semelhantes agrupados.                                                                                                                                                                                                                                                                                                \\

		\hline
		AAS Cliente
		 & O AAS Cliente é a parte que irá consumir as informações disponibilizadas pelo AAS Servidor. O cliente representa cada uma das partes envolvidas na cadeia de suprimentos. Pode representar uma instituição, uma pessoa física ou até mesmo uma outra máquina/produto.                                                                                                                                                                                                                                                                                               \\

		\hline
		AAS Repositório
		 & O repositório é o elemento que recebe, armazena e disponibiliza informações de descrição sobre todos os serviços disponíveis no mundo conectado, as descrições de serviços são disponibilizados em forma de contratos. O AAS recebe operações de ``publicação'' por parte do AAS Servidor e operações de ``busca'' por parte do AAS Cliente. O Repositório não atua como canal de comunicação entre AAS Cliente e Servidor, mas apenas fornece informações necessárias para que ambos os AAS possam se comunicar diretamente por meio da operação de ``interação''. \\

		\hline
	\end{tabular}
	\caption{Componentes da arquitetura para a I4.0.}
	\label{tab:componentes-ws}
\end{table}

Neste modelo, o contrato contendo a descrição dos serviços disponíveis nos submodelos de cada AAS é armazenado em um repositório comum, onde todos os AAS disponíveis no mundo conectado na I4.0 podem se tornar visíveis. A função do repositório é armazenar uma lista de contratos dos AAS disponíveis, mas não o providenciar o serviço em si. O serviço é fornecido pelo próprio AAS Servidor que o tornou público, servindo o repositório apenas como um elemento para a descoberta de serviços.

É importante notar que no mundo da I4.0 todo ativo é englobado por um AAS e se torna um Componente I4.0. Como o repositório detém informações e funções que agregam valor ao negócio, este pode também ser considerado um ativo e, portanto, possui o seu próprio AAS, que é responsável por toda a parte virtual deste ativo.

Cada AAS pode atuar tanto como um fornecedor de serviços (servidor), quanto como um solicitante de serviços (cliente), ou como ambos. Sempre usando o repositório como meio para a publicação ou busca dos serviços.

\subsection{Operações}

As operações de serviços de AAS e suas inter-relações com os componentes são mostradas por meio dos arcos na \autoref{fig:aas-ws} e suas descrições detalhadas são apresentadas na \autoref{tab:operacoes-ws}

\begin{table}[htb]
	\centering
	\begin{tabular}{p{3cm}p{12cm}}
		\hline
		\textbf{Operação}
		 & \textbf{Descrição}                                                                                                                                                                                                                                                                                                                                                                                                                                           \\

		\hline
		Publicação
		 & Ação tomada pelo AAS Servidor sempre que este componente queira anunciar um serviço para que possa ser descoberto por outros AAS. Nesta operação, o AAS Servidor envia o contrato descrevendo os serviços ofertados e a descrição de cada um desses serviços. Esta lista é recebida e armazenada pelo AAS Repositório, que a disponibiliza para acesso público.                                                                                              \\

		\hline
		Busca
		 & Ação tomada pelo AAS Cliente sempre que este precise consultar serviços de seu interesse. Nesta operação o AAS Cliente faz uma solicitação ao AAS Repositório com os parâmetros que definem o tipo e as restrições do serviço desejado. A operação de busca engloba também o fluxo contrário de informações, que é o envio da resposta da solicitação do AAS Repositório para o AAS Cliente.                                                                 \\

		\hline
		Interação
		 & Ação tomada pelo AAS Cliente sempre que este deseja invocar um serviço. O AAS Cliente estabelece uma conexão direta com o AAS Servidor e consome o determinado serviço solicitado. A operação de interação normalmente é feita após o recebimento da lista de contratos por parte do AAS Repositório, porém a interação pode ser feita diretamente caso o AAS Cliente já possua informações necessárias para o estabelecimento da conexão em \textit{cache}. \\

		\hline
	\end{tabular}
	\caption{Operações do WS para a I4.0.}
	\label{tab:operacoes-ws}
\end{table}

Para cada uma das operações deve ser definido o contrato, documento o qual descreve as funcionalidades do \textit{Web Service} e estabelece os padrões de comunicação suportados pelo AAS Servidor como, por exemplo, o padrão HTTP REST, HTTP SOAP, gRPC, etc; e especifica como acessar e quais são as operações ou métodos que estão disponíveis no serviço.

Quando o AAS atua como Servidor, este publica seu contrato no repositório por meio de uma API (\textit{Application Programming Interface}). Quando como Cliente, o AAS busca no repositório um serviço desejado e recebe uma lista de opções de serviços com suas respectivas descrições (contidas no contrato). Assim, o serviço mais adequado pode ser selecionado.

Uma vez definido o serviço a ser consumido, o AAS Cliente estabelecerá a conexão direta com o AAS Servidor por meio de algum dos padrões suportados, utilizando os detalhes contidos no contrato para localizar, contactar e invocar o serviço.

A \autoref{fig:pfs-ws} apresenta um diagrama PFS (\textit{Production Flow Schema}) (vide \autoref{sec:modelagem}), com o fluxo de ocorrência das operações básicas no WS para a I4.0.

\begin{figure}[htb]
	\centering
	\includegraphics[width=1\textwidth]{pfs-ws}
	\caption{Diagrama PFS das operações do WS.}
	\label{fig:pfs-ws}
\end{figure}

Os serviços fornecidos por um AAS são diversos. Entretanto, neste trabalho serão tratados com ênfase aqueles serviços que têm como objetivo o compartilhamento de informações sobre o ativo que possam agregar valor ao produto ao longo de sua cadeia de suprimentos. Ou seja, os serviços que extraem informações da MDP do AAS e as fornecem, mediante autenticação, às partes solicitantes ao longo da cadeia de suprimentos.

\section{Estrutura do AAS}

O conceito de Memória Digital do Produto (MDP) deve ser inserido na Indústria 4.0 com o objetivo de se agregar valor ao produto por meio da possibilidade de acesso a informações sobre o ativo entre parceiros ao longo da cadeia de suprimentos.

Nesta seção são apresentados os detalhes sobre uma possível estruturação do AAS contendo todas as partes necessárias (incluindo a MDP) para a implementação do compartilhamento de informações por meio de WSs.

A estrutura proposta do AAS é baseada em \citeonline{bader2019aas}, que estabelece a divisão do AAS em submodelos e o divide em duas partes: o cabeçalho (\textit{header}) e o corpo (\textit{body}).

O cabeçalho na estrutura proposta terá a função de providenciar informações públicas sobre o ativo que o identifiquem minimamente e que forneça uma descrição sobre seus serviços oferecidos. O cabeçalho deverá conter informações que podem ser acessadas sem a necessidade de autenticação como, por exemplo, seu identificador único universal (UUID - \textit{Universal Unique IDentifier}), o modelo e fabricante do ativo. O cabeçalho deverá conter também o contrato daquele \textit{Web Service} com as descrições de seus serviços públicos disponíveis.

O cabeçalho não terá a função de fornecer uma ficha técnica detalhada, mas apenas uma caracterização abstrata das funções/serviços do ativo. O cabeçalho deverá necessariamente conter o UUID do AAS. Sem o UUID o AAS Servidor se torna inacessível para qualquer uma das partes da cadeia de suprimentos.

Dentro dos moldes da estrutura proposta, o corpo (\textit{body}) de um AAS fornece as informações e funcionalidades sensíveis sobre o ativo, que podem ser acessadas mediante autenticação. As funcionalidades dos ativos são agrupadas em forma de submodelos, conforme estabelecido em \citeonline{bader2019aas, adolph2018roadmap, bedenbender2017aasexamples}, que são unidades de agrupamento de propriedades semelhantes. Os dados do ativo são armazenados nos próprios submodelos, enquanto a MDP (que também está contida no corpo do AAS) extrai e gerencia as informações dos submodelos de forma a estruturá-las para serem diretamente fornecidas ao AAS Cliente por meio de \textit{Web Services}.

A MDP no contexto de um AAS é uma \textit{view}, ou seja, ela agrega e espelha as informações referentes a cada um dos submodelos de um AAS e as organiza de forma a poderem ser facilmente disponibilizadas por meio de WSs.

Como a MDP é parte integral do AAS, que representa a parte virtual do ativo, esta pode ser fornecida em qualquer meio digital, inclusive em armazenamentos remotos em plataformas na nuvem. Estas plataformas específicas suportam o armazenamento de grandes quantidades de dados, assim como podem assegurar uma alta capacidade de processamento de requisições de serviços.

O corpo do AAS representa a carga útil (\textit{payload}) do AAS, pois é a porção de informação que é de fato relevante para o cliente que consumirá os serviços ofertados.

A estrutura de um AAS compatível com a arquitetura orientada a serviços proposta é apresentada na \autoref{fig:estrutura-aas}.

\begin{figure}[htb]
	\centering
	\includegraphics[width=0.7\textwidth]{estrutura-aas}
	\caption{Estrutura do AAS com seus submodelos e a MDP.}
	\label{fig:estrutura-aas}
\end{figure}

Os dados puros estão contidos nos submodelos. A MDP fornece apenas uma interface para a visualização dos dados dos submodelos. A MDP atua como um ponto único de extração de dados dos submodelos, evitando desta forma a extração de dados não tratados diretamente dos submodelos.

A MDP é capaz de realizar o gerenciamento dos dados, estabelecendo, por exemplo, políticas de acesso e/ou escrita e demais regras de negócio do AAS Servidor. Uma analogia é de que os submodelos representam o banco de dados, enquanto a MDP é a API de interface para operações de criação, leitura, atualização e exclusão (operações CRUD - \textit{Create, Read, Update, Delete}) nesta base de dados que são os submodelos.

Neste trabalho é dado foco aos submodelos que se relacionam a informações sobre o produto que sejam de interesse às partes ao longo da cadeia de suprimentos e que possam ser lidos ou escritos por meio dos WSs. Alguns exemplos desse tipo de submodelo podem incluir: a ficha técnica detalhada do ativo, submodelos de histórico de leitura de sensores, histórico de leitura de geolocalização (GPS), histórico de padrões de uso, etc. A relevância sobre cada uma dessas informações a serem armazenadas pela MDP e os impactos do amplo compartilhamento dessas informações ao longo da cadeia de suprimentos é discutido no \autoref{cha:ciclo-de-vida}.

\section{Os dados de serviços no repositório}

O repositório armazena as descrições dos serviços definidas pelos contratos de cada \textit{Web Service}, que são publicadas por diversos AAS-Servidores no mundo conectado da Indústria 4.0.

Os dados dos contratos contidos no repositório devem ser padronizados para garantir uma interpretação consistente destes por parte do AAS-Cliente, que consome os serviços. Para isso, metamodelos devem ser estabelecidos a fim de se criar moldes sobre os quais os dados devem ser estruturados.

\citeonline{bader2019aas} estabelece padrões de metamodelos para a implementação de submodelos no AAS, porém não aborda um possível repositório de serviços e o armazenamento de dados de serviços. A \autoref{tab:mdp-repositorio} traz uma proposta de metamodelo para a estruturação dos dados do repositório.

\begin{table}[htb]
	\centering
	\begin{tabular}{p{4cm}p{11cm}}
		\hline
		\textbf{Propriedade}
		 & \textbf{Descrição}                                                                                                                                                                                                                                                                                                                                                                                                                                                                       \\

		\hline
		ID do serviço
		 & Número do serviço para a sua identificação única entre todos os repositórios. O ID do serviço pode ser derivado do próprio ID do AAS-Servidor correspondente (E.g., ID\_SERVIDOR. ID\_SERVIÇO\_001).                                                                                                                                                                                                                                                                                     \\

		\hline
		Descrição do serviço
		 & Descrição sobre todos os atributos e funcionalidades do \textit{Web Service} juntamente com o tipo de resposta a ser dado em caso de sucesso ou falha. As descrições dos serviços descrevem todos os métodos que um determinado servidor suporta.                                                                                                                                                                                                                                        \\

		\hline
		ID do AAS Servidor correspondente ao serviço
		 & Tem a função de distinguir exclusivamente os AAS provedores de serviços e todos seus elementos \cite{adolphs2016structure} no mundo conectado da I4.0. Alguns tipos possíveis de identificadores são \cite{bader2019aas}: IRDI, IRI e UUID. O ID do AAS servidor permite que o consumidor de serviços possa localizar e invocar diretamente o serviço ofertado.                                                                                                                         \\

		\hline
		Descrição do AAS Servidor
		 & Breve descrição sobre o AAS Servidor como nome, modelo e companhia fabricante do ativo.                                                                                                                                                                                                                                                                                                                                                                                                  \\

		\hline
		\textit{Timestamp} do serviço
		 & Registros de data e hora relacionados a operações de inserção e atualização de informações sobre o serviço no repositório.                                                                                                                                                                                                                                                                                                                                                               \\

		\hline
		Qualidade do Serviço
		 & A métrica de qualidade de serviço (\textit{Quality of Service} - QoS) fornece indicadores sobre a qualidade do serviço prestado por um determinado AAS. O tempo médio de resposta do serviço baseado no tempo de resposta observado por diversas requisições executadas e a disponibilidade do AAS quando solicitado são índices que compõem QoS. Um índice para serviços de qualidade mais subjetiva pode ser criado baseado em avaliações de AAS Clientes que já consumiram o serviço. \\

		\hline
	\end{tabular}
	\caption{Metamodelo de dados do repositório.}
	\label{tab:mdp-repositorio}
\end{table}

A \autoref{tab:mdp-repositorio} é uma lista não exaustiva das propriedades necessárias para o armazenamento de um serviço. Ela apenas apresenta alguns tipos de propriedades necessárias para identificação e invocação de um serviço na rede.

Para APIs REST o Swagger (ou OpenAPI) é comumente utilizado como formato de contrato, detalhando todos os recursos e operações em um formato legível por humanos e por máquina. Para APIs SOAP, o WSDL é comumente utilizado. APIs com servidores gRPC suportam o contrato por meio de arquivos protobuf (\textit{protocol buffers}).

O AAS-Cliente na operação de busca fará uma requisição ao repositório ou a uma lista de repositórios e receberá (de cada um dos repositórios), por meio de uma API, a lista de serviços com uma estrutura de dados contendo os atributos chave-valor mencionados na \autoref{tab:mdp-repositorio}.

O AAS-Servidor será, portanto, o responsável pela geração e atualização dos dados referentes a seus serviços no AAS-Repositório. Os dados relacionados a qualidade de serviço (QoS), entretanto, devem ser gerados pelo próprio AAS Repositório para garantir a integridade dos índices.

A inserção e atualização de um registro no AAS-Repositório deve acontecer sempre que um AAS-Servidor é instalado ou atualizado, o que acarreta uma nova operação de publicação.

\section{Fluxo de fornecimento de serviços}

As etapas para o fornecimento de serviços na I4.0 segue um fluxo padrão. Os submodelos agregam informações semelhantes que podem ser lidas ou escritas (via MDP) e disponibilizada por meio de \textit{Web Services} a qualquer uma das partes ao longo da CS mediante autenticação.

Um fluxo de leitura/escrita de dados pode ser exemplificado com uma CS simples contendo três membros: um fabricante, um distribuidor e um consumidor; cada membro da CS é um AAS Cliente diferente. O fabricante cria o produto, que será o AAS Servidor, e define a estrutura do AAS com os submodelos necessários. A título de exemplo, três submodelos são definidos: submodelo ``Geolocalização'', submodelo ``Sensores'' e submodelo ``Documentação''. Ao longo do ciclo de vida do produto, os membros da CS (fabricante, distribuidor e consumidor) podem interagir com esses submodelos, fazendo sua leitura para o caso dos submodelos de geolocalização e sensores, e podendo fazer a leitura e/ou escrita para o caso do submodelo de documentação.

A \autoref{fig:webservice-multielo} demonstra este cenário mencionado com o fluxo de operações básicas do WS em funcionamento. Neste exemplo, o (\textbf{a}) AAS de um produto mantém contato com o (\textbf{b}) AAS da empresa do fabricante, com o (\textbf{c}) AAS da empresa do distribuidor, e com o (\textbf{d}) AAS do consumidor final, fornecendo o serviço de consulta de informações de diferentes submodelos para cada um dos solicitantes.

\begin{figure}[htb]
	\centering
	\includegraphics[width=0.75\textwidth]{webservice-multielo}
	\caption{Exemplificação das operações de publicação e busca com múltiplos clientes.}
	\label{fig:webservice-multielo}
\end{figure}

Na \autoref{fig:webservice-multielo} são mostrados os três submodelos do AAS Produto: (\textbf{e}) Submodelo ``geolocalização'', (\textbf{f}) submodelo ``sensores'' e (\textbf{g}) submodelo ``documentação''. A (\textbf{h}) MDP realiza o gerenciamento dos dados de todos os submodelos e os disponibiliza para acesso pelos serviços, as descrições dos serviços são armazenados em forma de um (\textbf{i}) contrato. Este contrato é (\textbf{j}) publicado no repositório.

O (\textbf{k}) AAS Repositório recebe o contrato do AAS Produto e o disponibiliza para consulta. O AAS Repositório receberá também os contratos de diversos outros AAS (não representado na figura).

Os AAS Clientes fazem a (\textbf{l}) busca no AAS Repositório. As buscas são feitas com parâmetros a fim de se restringir qual tipo de serviço aquele cliente pretende consumir, podendo-se restringir a busca, inclusive, ao serviço de um AAS específico, identificando-o por meio de seu UUID.

Cada AAS Cliente (Fabricante, Distribuidor ou Consumidor), portanto, realiza a consulta ao AAS Repositório com os parâmetros de interesse e recebe a resposta de contratos contendo descrições detalhadas sobre os serviços disponíveis e informações para localizar, contactar e invocar estes serviços.

O próximo passo (após o recebimento da resposta do AAS Repositório) é a decisão interna de cada AAS Cliente sobre qual serviço selecionar. Uma vez definido, o AAS Cliente estabelece uma comunicação direta com o AAS Servidor (produto) para o consumo do serviço selecionado, a interação. Os passos de seleção do serviço específico e interação não são demonstrados na figura.

Este é um exemplo de consulta única. Em aplicações reais, o cliente normalmente invocaria o serviço de diversos AAS Servidores ao mesmo tempo, como, por exemplo, um fabricante solicitando informações de todas as máquinas de um modelo específico que foram vendidas a clientes espalhados pelo mundo para realizar análise de dados a fim de se fazer uma manutenção preditiva por meio da identificação de potenciais falhas. Tal cenário e demonstrado em forma de um segundo exemplo na \autoref{fig:webservice-multiproduto}.

\begin{figure}[htb]
	\centering
	\includegraphics[width=1\textwidth]{webservice-multiproduto}
	\caption{Exemplificação das operações de publicação e busca com múltiplos produtos.}
	\label{fig:webservice-multiproduto}
\end{figure}

A \autoref{fig:webservice-multiproduto} demonstra a situação de uma consulta de um AAS Cliente em múltiplos produtos. Neste exemplo, cada AAS Produto (\textbf{a}, \textbf{b} e \textbf{c}) realiza uma operação de (\textbf{f}) publicação no (\textbf{e}) AAS Repositório.

O (\textbf{d}) AAS Fabricante (Cliente) por sua vez faz uma busca no AAS Repositório especificando os parâmetros que restrinjam a pesquisa a somente determinados modelos de produtos, e recebe como resposta todas as descrições dos serviços que correspondem aos critérios de busca.

\section{Mapeamento das operações no RAMI4.0}

Segundo \citeonline{iec2017rami}, o RAMI4.0 fornece uma visão estruturada dos principais elementos de um ativo usando um modelo de níveis composto por três eixos. Desta forma, inter-relações complexas podem ser divididas em seções menores e mais gerenciáveis, combinando os três eixos para representar cada aspecto relevante do estado do ativo em cada ponto de seu ciclo de vida.

Esta seção tem o objetivo de mapear as operações envolvidas nos fluxos de compartilhamento informações ao longo da cadeia de suprimentos para as camadas do RAMI4.0 de forma a representar as etapas relacionadas ao trânsito de informações em um modelo unificado.

O mapeamento para o RAMI4.0, que é uma arquitetura de referência para a I4.0, contribui para facilitar a execução de implementações de conceitos de I4.0 uma vez que estabelece um padrão de arquitetura que deve ser adotado por todos, garantindo a interoperabilidades entre os sistemas.

Até então neste capítulo os componentes da arquitetura (cliente, servidor e repositório) foram referenciados como AAS, sendo esta somente a parte virtual de um componente. A partir desta seção os ativos reais passam a compor o fluxo de compartilhamento de informações e, portanto, os termos C4.0-Cliente, C4.0-Servidor e C4.0-Repositório passarão a ser utilizados para denotar os componentes completos, ou seja a junção do AAS com sua parte real (o ativo).

\subsection{Funcionalidades das camadas do RAMI4.0}

Na \autoref{sub:rami4} foram apresentados os detalhes do RAMI4.0 e o detalhamento de cada nível do eixo Camadas com suas funções genéricas. Nesta subseção são mostradas as funções específicas da arquitetura proposta para cada camada o sob o contexto do compartilhamento de informações pela CS .

Na camada \textbf{Ativo} estará os produtos (servidor de informações), clientes (empresas, pessoas, consumidores das informações) e repositórios (armazenamento de contratos).

Para o compartilhamento de informações do produto no mundo I4.0, os dados a serem extraídos do ativo são estrategicamente selecionados com o objetivo de reunir somente os que agreguem valor ao próprio ativo. Assim, estes dados selecionados são extraídos do ativo e repassados às camadas superiores para que possam ser armazenados e compartilhados por meio de serviços.

A camada \textbf{Integração} está presente no C4.0-Servidor, pois dele serão extraídos os dados desde o ativo até as camadas superiores. Tanto o C4.0-Cliente quanto o C4.0-Repositório operam primariamente nas camadas virtuais, porém também possuem esta camada para outros fluxos de integração com os ativos da empresa, seja ele um \textit{software} (e.g., uma base de dados de ERP, um repositório) ou um componente físico.

Com relação à camada \textbf{Comunicação}, para a arquitetura de compartilhamento de informações de ativos, não haverá comunicação entre C4.0s dentro da mesma empresa uma vez que todas as operações de um WS (publicação, busca e interação) ocorrem entre componentes de organizações distintas. Esta camada, entretanto, é necessária para as diversas outras comunicações inter-organizacionais. Os protocolos estabelecidos nesta camada para a integração vertical são independentes dos protocolos para a integração horizontal, que são definidos na camada funcional.

Na camada \textbf{Informação}  do C4.0-Repositório estarão armazenados os contratos disponibilizados por cada produto. No C4.0-Servidor, esta camada armazena os submodelos e a MDP.

Na camada \textbf{Funcional} ocorre toda a integração horizontal entre as partes da cadeia de suprimentos de um produto. Os serviços são disponibilizados por meio da camada funcional, portanto esta é a interface entre os C4.0s de diferentes empresas.

O fornecimento de serviços ao longo da cadeia de suprimentos é considerado uma integração horizontal uma vez que cada parte da CS representa uma organização diferente. Para o fornecimento e consumo desses serviços de compartilhamento de informações, devem ser definidas também as especificações da API, ou seja, o padrão de requisição e resposta para o fornecimento e consumo de serviços como, por exemplo, o padrão REST.

Na camada \textbf{Regra de Negócio} estarão as restrições aplicadas sobre os \textit{Web Services}, como as políticas de privacidade de dados (e.g., restrições de acesso a determinados serviços) e as restrições legais de cada país.

A regra de negócio estabelecerá quem na cadeia de suprimentos terá permissão para acessar quais informações do produto e quando. Um fabricante, por exemplo, terá acesso aos dados de padrões de uso de um ativo somente sob a permissão do cliente, o que representa uma regra nesta camada. Um distribuidor, por sua vez, só poderá ter acesso à localização do ativo enquanto o produto estiver sob sua custódia.

Para cada tipo de operação relacionada a um Componente I4.0, é necessário detalhar o fluxo de dados e de eventos acontecendo em cada uma das camadas. Este detalhamento permite que implementações de soluções I4.0 sejam facilitadas e garante que a criação dessas soluções por diversos desenvolvedores de sistemas resulte em sistemas que sejam interoperáveis, independentemente da tecnologia adotada.

O Componente I4.0 pode ainda ser mais detalhadamente especificado, identificando se o componente representa um produto em desenvolvimento ou um produto em produção. Estes detalhes são cobertos pelo eixo ``Ciclo de Vida e Cadeia de Valor'' e considerações sobre este eixo envolvendo a arquitetura proposta baseada em WSs é apresentada no \autoref{cha:ciclo-de-vida}.

A \autoref{fig:webservice-rami} apresenta os componentes da arquitetura de fornecimento de WSs dentro do eixo Camadas do RAMI4.0.%, que são:

%	\begin{itemize}
%		\item \textbf{Ativo}: Pessoas e empresas (cliente), produtos (servidor) e contratos (repositório); 
%		\item \textbf{Integração}: Virtualização das informações, protocolos de transferência de dados (Ethernet, 5G, Wi-Fi, etc); 
%		\item \textbf{Comunicação}: Protocolos de comunicação vertical (OPC UA); 
%		\item \textbf{Informação}: Controle de acesso / autenticação, análise de dados, armazenamento, descrição dos serviços (virtual);
%		\item \textbf{Funcional}: Serviços de compartilhamento de informações, protocolos de comunicação horizontal (HTTPS, etc), interface horizontal entre AAS; 
%		\item \textbf{Regra de negócio}: Restrições legais, políticas de privacidade.
%	\end{itemize}

\begin{figure}[H]
	\centering
	\includegraphics[width=0.7\textwidth]{webservice-rami}
	\caption{Camadas do RAMI4.0 com os elementos da arquitetura.}
	\label{fig:webservice-rami}
\end{figure}

\subsection{Operação de Publicação}

A \autoref{fig:rami-publicacao} apresenta diagramas PFS do fluxo de atividades para a operação de publicação de um C4.0-Servidor em um C4.0-Repositório.

A operação de publicação se inicia sempre que um novo ativo e seu AAS correspondente for instalado ou quando esse AAS for atualizado.

\begin{figure}[htb]
	\centering
	\includegraphics[width=1\textwidth]{rami-publicacao}
	\caption{Diagrama PFS da operação de publicação.}
	\label{fig:rami-publicacao}
\end{figure}

Esta operação é iniciada pelo C4.0-Servidor recém instalado ou atualizado, seguindo um fluxo de atividades estabelecido até chegar no ativo do C4.0-Repositório, os passos são detalhados a seguir:

\begin{enumerate}

	\item Os serviços de compartilhamento de informações do C4.0-Servidor disponíveis são identificados e listados.

	\item As restrições legais e as políticas de privacidade para um determinado C4.0 são acessadas. Esta atividade adicionará restrições aos serviços a serem publicados. Estas restrições serão incorporadas à descrição de cada serviço a ser publicado.

	\item O contrato contendo as descrições de serviços disponíveis no C4.0-Servidor é disponibilizado.

	\item O contrato é enviado ao C4.0-Repositório via API.

	\item O C4.0-Repositório recebe o contrato no formato de intercâmbio definido. A descrição dos serviços nesta fase já contém todas as informações para a identificação do serviço e de seu componente correspondente.

	\item Os dados recebidos alimentam uma interface para comunicação com o ativo real.

	\item O contrato com a lista das descrições dos serviços é armazenado junto aos demais contratos no ativo ``repositório'' do C4.0-Repositório.

\end{enumerate}

\subsection{Operação de Busca}

A operação de busca é dividida em duas partes: a requisição e a resposta. A requisição é a iniciativa do C4.0-Cliente de requerer a lista de contratos contidas em um C4.0-Repositório, que descreve os serviços de cada ativo cadastrado. O fluxo de atividades da requisição em uma operação de busca é apresentado na \autoref{fig:rami-busca-requisicao}.

Já a resposta da requisição em uma operação de busca é feita pelo C4.0-Repositório para o C4.0-Cliente e é apresentada na \autoref{fig:rami-busca-resposta}.

\begin{figure}[htb]
	\centering
	\includegraphics[width=1\textwidth]{rami-busca-requisicao}
	\caption{Diagrama PFS da requisição em uma operação de busca.}
	\label{fig:rami-busca-requisicao}
\end{figure}

A operação de requisição para a busca de um serviço parte do C4.0-Cliente e segue um fluxo estabelecido até chegar ao C4.0-Repositório, esses passos são detalhados a seguir:

\begin{enumerate}

	\item O processo de requisição se inicia com a definição por parte do ativo do C4.0-Cliente (pessoa ou empresa) sobre qual informação se deseja consultar como, por exemplo, leituras de sensores, localização geográfica, manuais, etc.

	\item A partir do tipo de informação a ser consultada, define-se os parâmetros de consulta, que representam o conjunto de restrições que estabelecem qual é exatamente o tipo de serviço que o AAS Cliente deseja consumir. Para os serviços que visam a extração de informações do ativo, os parâmetros representam, por exemplo, o ID do provedor de serviços, o horário e data de um determinado evento, uma filtragem por modelos específicos de um produto, etc.

	\item Os parâmetros de consulta alimentam uma interface para que a solicitação possa ser virtualizada e integrada ao AAS. Nesta atividade a intenção de solicitação de um serviço é virtualizada.

	\item Opcionalmente, são armazenados os detalhes da solicitação em um registro de solicitações.

	\item A requisição é enviada ao C4.0-Repositório via API.

	\item O C4.0-Repositório recebe a solicitação e a insere ao final da lista de solicitações para ser processada.

	\item A requisição é processada. Identifica-se nesta atividade se a requisição é válida e se ela contém todos os parâmetros necessários para a consulta.

	\item Estabelece-se uma interface para a interação entre o AAS e seu ativo real para a consulta das informações solicitadas na requisição.

	\item Realiza-se a leitura da lista de contratos no repositório utilizando os parâmetros de consulta estabelecidos.

\end{enumerate}

Após a requisição, o C4.0-Repositório envia a resposta ao C4.0-Cliente. O fluxo de atividades da resposta é apresentada em diagramas PFS na \autoref{fig:rami-busca-resposta}.

\begin{figure}[htb]
	\centering
	\includegraphics[width=1\textwidth]{rami-busca-resposta}
	\caption{Diagrama PFS da resposta em uma operação de busca.}
	\label{fig:rami-busca-resposta}
\end{figure}

Detalhadamente, a resposta do C4.0-Repositório segue o fluxo de atividades, começando pela disponibilização das informações reais do ativo repositório:

\begin{enumerate}
	\item Os dados do ativo são disponibilizados para as camadas superiores.

	\item Os dados disponibilizados são virtualizados para a integração com o AAS.

	\item O contrato contendo a descrições dos serviços é disponibilizado para acesso por parte da camada Funcional. A lista  pode conter serviços válidos, assim como pode conter mensagens de erro por conta de solicitações inválidas ou buscas retornando zero correspondências.

	\item A resposta é enviada ao C4.0-Cliente via API.

	\item O C4.0-Cliente recebe a resposta contendo os contratos no formato de intercâmbio definido.

	\item Após a recepção da lista de contratos contendo todos os serviços disponíveis, é feito o processamento para a definição do serviço mais adequado. Esta fase geralmente não fornece múltiplas opções para a consulta de serviços de compartilhamento de informações ao longo da cadeia de suprimentos uma vez que os parâmetros de consulta na requisição ao repositório geralmente já delimitam exatamente o serviço e o AAS-Servidor que o cliente busca.
\end{enumerate}

\subsection{Operação de Interação}

A operação de interação é a fase final para o consumo de um serviço disponibilizado no mundo conectado da I4.0. Assim como a busca, a interação é divida em requisição e resposta. Primeiramente, o C4.0-Cliente faz uma requisição de consumo de um serviço com parâmetros e então a requisição é processada e respondida pelo C4.0-Servidor.

A \autoref{fig:rami-interacao-requisicao} apresenta o fluxo de atividades em uma requisição de um serviço e a \autoref{fig:rami-interacao-resposta} a resposta do servidor.

\begin{figure}[htb]
	\centering
	\includegraphics[width=1\textwidth]{rami-interacao-requisicao}
	\caption{Diagrama PFS da requisição de um serviço em uma operação de interação.}
	\label{fig:rami-interacao-requisicao}
\end{figure}

A requisição de um serviço na operação de interação é iniciada pelo C4.0-Cliente e é enviada diretamente ao C4.0-Servidor usando o contrato fornecido pelo C4.0-Repositório. O fluxo de atividades para a requisição de um serviço é detalhada a seguir:

\begin{enumerate}

	\item Surge a necessidade de consumo de uma informação por parte do ativo.

	\item Define-se qual é o serviço mais adequado e de qual C4.0 será consumido.

	\item As restrições legais e as políticas de privacidade do C4.0-Cliente são consultadas. Essas regras de negócio são incorporadas ao processamento sobre a definição do serviço mais adequado a ser escolhido.

	\item A requisição do serviço é enviada ao C4.0-Servidor via API.

	\item O C4.0-Servidor recebe a requisição e a insere ao final da lista de requisições para ser processada.

	\item É feita a autenticação da identidade do C4.0-Cliente e a autorização para o consumo do serviço. Nesta atividade é verificado se o cliente possui autorização para consumir o serviço e consequentemente os dados que estão sendo solicitados.

	\item Consulta-se as restrições legais e as políticas de privacidade do C4.0-Servidor para a autorização ou bloqueio do fornecimento do serviço ao cliente solicitante.

	\item A requisição é processada. Identifica-se nesta atividade se a requisição é válida e se ela contém todos os parâmetros necessários para o fornecimento do serviço.

\end{enumerate}

Após o recebimento e processamento da requisição de um serviço, o C4.0-Servidor deve fazer a extração e envio das informações do ativo. A resposta contendo as informações sobre o produto começa com a emissão de um evento no ativo ou, caso a informação já esteja disponível na MDP, começa direto da camada de Informação. Esta informação percorre um fluxo padrão para que seja disponibilizada ao C4.0-Cliente por meio do serviço. A \autoref{fig:rami-interacao-resposta} mostra as atividades do fluxo de resposta.

\begin{figure}[htb]
	\centering
	\includegraphics[width=1\textwidth]{rami-interacao-resposta}
	\caption{Diagrama PFS da resposta de um servidor ao se solicitar um serviço.}
	\label{fig:rami-interacao-resposta}
\end{figure}

O detalhamento do fluxo de atividades da resposta é detalhado a seguir:

\begin{enumerate}

	\item Um evento físico no mundo real é emitido.

	\item O evento fornece sinais físicos que podem ser medidos.

	\item Os sinais físicos são interpretados e virtualizados. Nesta atividade é criado um correspondente virtual para o evento do ativo físico, ou seja, os dados são digitalizados e disponibilizados ao AAS.

	\item É definido o meio de transporte e o procolo de transferência de dados como, por exemplo, o Wi-Fi, Ethernet, 5G, etc.

	\item Os dados são devidamente transportados pelo meio e protocolo definidos até uma central de processamento.

	\item Os novos dados sobre o ativo são armazenados pela MDP nos submodelos junto aos demais dados já existentes.

	\item Os dados são disponibilizados ao serviço. Caso a informação já estivesse contida na MDP do C4.0-Servidor esta seria a primeira atividade da resposta uma vez que não seria necessária a extração dos dados diretamente do ativo.

	\item O serviço é executado e sua resposta é gerada. O serviço pode executar quaisquer operações sobre os dados atualizados sobre o ativo assim como sobre o histórico de registros antigos já disponíveis nos submodelos.

	\item A resposta do serviço (fornecimento do serviço) é enviada via API.

	\item O C4.0-Cliente recebe a resposta do serviço no formato de intercâmbio definido.

	\item Os dados recebidos na resposta são processados.

	\item As informações geradas alimentam uma interface para comunicação com o ativo real.

	\item As informações do C4.0-Servidor processadas são disponibilizadas ao ativo do C4.0-Cliente.
\end{enumerate}
