\chapter{Publicações decorrentes do trabalho}


	\textbf{Publicação 1}:
	\begin{itemize}
		\item \underline{Título do trabalho}: “Análise de implementação de IoT na cadeia logística”
		\item \underline{Congresso}: XXXIX Encontro Nacional de Engenharia de Produção - ENEGEP 2019
		\item \underline{Status}: Aprovado, apresentado e publicado nos anais do evento
		\item \underline{Autores}:  Henrique A. Vitoi, Fabrício Junqueira, Paulo E. Miyagi
		\item \underline{Apresentação}: 16 de outubro de 2019, Santos/SP
	\end{itemize}
	
	\bigskip

	\textbf{Publicação 2}:
	\begin{itemize}
		\item \underline{Título do trabalho}: “Big Data on Machine to Machine Integration's Requirement Analysis Within Industry 4.0”
		\item \underline{Congresso}: DoCEIS 2019: Technological Innovation for Industry and Service Systems
		\item \underline{Status}: Aprovado e publicado
		\item \underline{Autores}:  Felipe A. Coda, Rafael M. Salles, Henrique A. Vitoi, Marcosiris A. O. Pessoa, Lucas A. Moscato, Diolino J. Santos Filho, Fabrício Junqueira, Paulo E. Miyagi
		\item \underline{Publicação}: 16 de abril de 2019
	\end{itemize}

% ----------------------------------------------------------

\chapter{Cronograma detalhado}

	O cronograma planejado é mostrado na \autoref{tab:cronograma}.
	
	\begin{table}[htb]
		\centering
		\caption{Cronograma detalhado de atividades}
		\makebox[\textwidth][c]{\footnotesize
		\begin{tabular}{|c|c|c|c|c|c|c|c|c|c|c|c|c|}
			
			\hline 
			
			& \multicolumn{2}{c|}{2018}
			& \multicolumn{6}{c|}{2019}
			& \multicolumn{4}{c|}{2020} \\
			
			\hline
			Etapas
			& \begin{tabular}[c]{@{}c@{}}set/\\   out\end{tabular} 
			& \begin{tabular}[c]{@{}c@{}}nov/\\   dez\end{tabular} 
			& \begin{tabular}[c]{@{}c@{}}jan/\\   fev\end{tabular} 
			& \begin{tabular}[c]{@{}c@{}}mar/\\   abr\end{tabular} 
			& \begin{tabular}[c]{@{}c@{}}mai/\\   jun\end{tabular} 
			& \begin{tabular}[c]{@{}c@{}}jul/\\   ago\end{tabular} 
			& \begin{tabular}[c]{@{}c@{}}set/\\   out\end{tabular} 
			& \begin{tabular}[c]{@{}c@{}}nov/\\   dez\end{tabular} 
			& \begin{tabular}[c]{@{}c@{}}jan/\\   fev\end{tabular} 
			& \begin{tabular}[c]{@{}c@{}}mar/\\   abr\end{tabular} 
			& \begin{tabular}[c]{@{}c@{}}mai/\\   jun\end{tabular}
			& \begin{tabular}[c]{@{}c@{}}jul/\\   ago\end{tabular} \\ 
			
			\hline
			Cumprimento dos créditos 
			& \cellcolor[HTML]{9AFF99}C
			& \cellcolor[HTML]{9AFF99}C 
			& \cellcolor[HTML]{9AFF99}C 
			& \cellcolor[HTML]{9AFF99}C 
			& \cellcolor[HTML]{9AFF99}C 
			& \cellcolor[HTML]{9AFF99}C 
			&
			&
			&
			&
			& \\
			
			\hline
			Levantamento bibliográfico
			& \cellcolor[HTML]{9AFF99}C
			& \cellcolor[HTML]{9AFF99}C
			& \cellcolor[HTML]{9AFF99}C
			& \cellcolor[HTML]{9AFF99}C
			& \cellcolor[HTML]{9AFF99}C
			& \cellcolor[HTML]{9AFF99}C
			& \cellcolor[HTML]{9AFF99}C
			& \cellcolor[HTML]{9AFF99}C
			& \cellcolor[HTML]{FFCB2F}A
			& \cellcolor[HTML]{FFCB2F}A
			& \cellcolor[HTML]{FFCB2F}A
			& \cellcolor[HTML]{FFCB2F}A \\
			
			\hline
			Desenvolvimento do projeto
			&
			&
			& \cellcolor[HTML]{9AFF99}C
			& \cellcolor[HTML]{9AFF99}C
			& \cellcolor[HTML]{9AFF99}C
			& \cellcolor[HTML]{9AFF99}C
			& \cellcolor[HTML]{9AFF99}C
			& \cellcolor[HTML]{9AFF99}C
			& \cellcolor[HTML]{FFCB2F}A
			& \cellcolor[HTML]{FFCB2F}A
			& \cellcolor[HTML]{FFCB2F}A
			& \\ 
			
			\hline
			Exame de Qualificação
			&
			&
			&
			&
			&
			&
			&
			&
			&
			& \cellcolor[HTML]{FFCB2F}A
			&
			& \\
			
			\hline
			Defesa da dissertação
			&
			&
			&
			&
			&
			&
			&
			&
			&
			&
			&
			& \cellcolor[HTML]{FFCB2F}A \\
			\hline
			
		\end{tabular}}
		\label{tab:cronograma}
		\fonte{O autor.}
	\end{table}

	A data estipulada para defesa da dissertação pode ser adiada conforme necessidade para refinamento do projeto, adicionando-se mais meses para levantamento bibliográfico e desenvolvimento do projeto, respeitando de toda forma o prazo máximo para depósito da dissertação. 
	
	Disciplinas cursadas nos períodos 2018/3, 2019/1 e 2019/2:
	\begin{itemize}
		\item PMR5024 - Simulação de Sistemas;
		\item PTC5751 - Internet das coisas;
		\item PEA5003 - Sistemas Inteligentes de Transporte;
		\item PMR5023 - Modelagem e Análise de Sistemas;
		\item PTR5744 - Pesquisa Operacional;
		\item PRO5807 - Logística e Cadeia de Suprimentos;
		\item PMR5402 - Controle de Sistemas.
	\end{itemize}

	% Dizer no que as disciplinas ajudaram no trabalho. 
	


% ----------------------------------------------------------
\chapter{Conclusão}
\label{cha:conclusao}
	\lipsum[1-1]
	.