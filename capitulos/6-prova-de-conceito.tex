\chapter{Prova de conceito}

	Para o fornecimento e consumo de serviços entre AAS Clientes e Servidores, diversas protocolos e tecnologias podem ser adotadas.
	
	Este capítulo tem o objetivo de apresentar uma implementação funcional da arquitetura apresentada no \autoref{cha:arquitetura} como prova de conceito, adotando alguns protocolos e tecnologias atualmente comuns meio da engenharia de \textit{software} e desenvolvimento de sistemas.
	

\section{ Arquitetura do WS e tecnologias utilizadas }

	O protocolo de comunicação para o fornecimento de WSs mais comumente aplicado atualmente é o HTTP \cite{gruner2016restful}, seguindo as regras de operações padronizadas definidas pelo padrão REST.
	
	Alguns outros protocolos também são aplicados para oferecimentos de WSs, como o MQTT, que está presente principalmente na área de automação residencial e IoT \cite{yokotani2016mqtt}.


\section{ Estruturação dos dados da MDP }

	A estrutura proposta usa o padrão de troca de dados JSON, que utiliza texto legível a humanos, no formato atributo-valor (natureza auto-descritiva). O um modelo de transmissão de informações no formato JSON é muito usado em WSs que usam transferência de estado representacional (REST) e aplicações AJAX, substituindo o uso do XML.
	
	A estrutura de armazenamento implementada usa banco de dados orientado a documentos que usa documento em formato JSON com esquemas pré-definidos.
	
	A \autoref{fig:json} mostra um exemplo de estruturação de dados para troca e armazenamento de informações em JSON.
	
	\begin{figure}[htb]
		\centering
		\caption{Formato de intercâmbio de informações da MDP em JSON.}
		\label{fig:json}
		\includegraphics[width=0.8\textwidth]{json}
		\fonte{O autor.}
	\end{figure}

	
\section{ API de interação Cliente-Servidor }

	Para fins de escrita e pelo Cliente a fins de leitura é realizado por meio de uma API REST.
	
	A API REST é invocada como uma interface para acesso aos serviços de um AAS Servidor, podendo extrair dados internos de sua MDP e executar operações CRUD (criação, leitura, atualização e exclusão).
	